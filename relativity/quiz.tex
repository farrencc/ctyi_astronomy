\documentclass{article}
\usepackage{amsmath,amsfonts,amssymb,amsthm}
\usepackage{braket}
\usepackage{bbold}
\usepackage{physics}
\usepackage{graphicx}
\usepackage[utf8]{inputenc}
\usepackage[T1]{fontenc}
\usepackage{mathtools}
\usepackage[thinc]{esdiff}
\usepackage{bigints}
\usepackage{enumerate}
\usepackage{wasysym}
\usepackage{pythonhighlight}
\usepackage{caption}
\usepackage{subcaption}
\usepackage{esint}
\usepackage[top=1in,bottom,1in,left=1in,right=1in]{geometry}
\DeclareMathOperator{\arccosh}{arcCosh}
\DeclareMathOperator{\arcsinh}{arcsinh}
\DeclareMathOperator{\arctanh}{arctanh}
\DeclareMathOperator{\arcsech}{arcsech}
\DeclareMathOperator{\arccsch}{arcCsch}
\DeclareMathOperator{\arccoth}{arcCoth}
\title{CTYI Spring 2024 - Astronomy - Relativity Quiz}
\author{Cas}
\date{03 / 02 / 2024}
\begin{document}
\maketitle

\section*{Problem 1}
What Famous Scientist came up with the theory of relativity?
\begin{enumerate}
    \item Isaac Newton
    \item Albert Einstein
    \item Galileo Galilei
    \item Stephen Hawking
    \item Paul Dirac
\end{enumerate}
\section*{Problem 2}
What two principles make up the theory of special relativity?
\begin{enumerate}
    \item Principle of Nuclear Forces and the Principle of Constant Speed of Light
    \item Principle of Relativity and the Principle of mass
    \item Principle of mass and the principle of nuclear forces
    \item Principle of Relativity and the Principle of Constant Speed of Light
    \item None of the Above
\end{enumerate}
\section*{Problem 2}
According to the Theory of Relativity, the speed of light in a vacuum is \underline{\hspace{3cm}} for all observers?
\begin{enumerate}
    \item Different
    \item Dependent on the source of light
    \item Dependent on the observer
    \item The same
    \item None of the Above
\end{enumerate}
\section*{Problem 3}
True or False: The Theory of Relativity was used in the creation of the atomic bomb.
\begin{enumerate}
    \item True
    \item False
\end{enumerate}
\section*{Problem 4}
In the equation $E = mc^2$, what does $E$ stand for?
\begin{enumerate}
    \item Energy
    \item Entropy
    \item Electricity
    \item Electromagnetism
    \item None of the Above
\end{enumerate}
\section*{Problem 5}
In the equation $E = mc^2$, what does $m$ stand for?
\begin{enumerate}
    \item Movement
    \item Mana
    \item Mass
    \item Momentum
    \item None of the Above
\end{enumerate}
\section*{Problem 6}
In the equation $E = mc^2$, what does $c$ stand for?
\begin{enumerate}
    \item The speed of light
    \item The speed of sound
    \item The speed of electricity
    \item The speed of magnetism
    \item None of the Above
\end{enumerate}
\section*{Problem 7}
The faster an object is moving in relation to an observer, the \underline{\hspace{3cm}} it will appear to an observer
\begin{enumerate}
    \item Longer 
    \item Shorter
    \item Slower 
    \item Faster 
    \item the same 
\end{enumerate}
\section*{Problem 8}
True or False: The faster a spaceship is flying, the shorter it will appear to people \textit{on board the ship}.
\begin{enumerate}
    \item True
    \item False
\end{enumerate}
\section*{Problem 9}
If the mass of an object changes, what will also change?
\begin{enumerate}
    \item The speed of light
    \item The energy of the object
    \item The length of the object
    \item The time of the object
    \item None of the Above
\end{enumerate}


















\end{document}